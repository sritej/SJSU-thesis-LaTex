\chapter{Introduction}


JavaScript has become the most common web development language. Though once seen as a client side scripting language that only interacts with the user to control visual layout, today JavaScript is widely used to communicate asynchronously by sending and receiving data to and from the server thus giving web applications a much more interactive look. 

Developers generally build or develop websites by grabbing JavaScript code-snippets from various sources.  This integration might be an intentional one or might be a result of some malicious code being inserted due to some security vulnerability. This act of injecting code from different untrusted zones into a website is generally referred to as cross-site scripting attack (XSS). There are chances that this code might still breach several security policies no matter if it has been included either knowingly or unknowingly. This malicious code operates with the same rights as that of the normal code developed and written by the web developer, thus leading to different security issues. 

In this paper I worked on addressing one of such security issue which might leak some information to the attacker due to an unexpected error in the program's control flow.


\section{Security Challenges} 

There are a wide variety of security measures that have been implemented to safeguard against these problems. As the content is changing and is very dynamic in nature it has become very challenging to keep up with the best practices. One way is to build these security controls into the browser.

One way to tackle such security issues is to include proper information flow analysis within the browser. Advantages of this approach being that the users are protected even when they visit websites with no server side security. Though this guarantees a systematic solution against proven security attacks, it has failed to achieve its purpose in many of the cases as they only concentrate on static information flow type systems. Here lies the challenge as JavaScript is dynamically typed, only dynamic information flow analysis is well suited to achieve protection against these malicious scripts.


\section{Dynamic Information Flow}

JavaScript is a dynamically typed language. It is used to embed within web pages and is executed by the browser.  It is now-a-days the most widely used programming language of all the current web 2.0 applications ~\cite{JavaScript}. It is generally used for client side validations, password fields check, major websites like search engines and mapping applications. JavaScript code from multiple sources executes with the same authority as that of an authorized user. To help address these sorts security issues, we investigate the methodology of tracking the data/information flow dynamically during runtime. There has been some amount of work done prior to this on type systems ~\cite{Javastatic1}, ~\cite{Javastatic3}, but most of them are not suitable to this kind of dynamic languages. Further, having just a static analysis approach can be problematic when used with in the browser.

Different ideas on dynamic information flow analysis has been published in previous papers ~\cite{harden}, ~\cite{basa}, ~\cite{bib3} where the main discussion was around a special type of value called {\it faceted value}. Faceted value approach was seen as one of the good way of achieving multi-process execution with the efficiency of single-process execution ~\cite{multiproc2} ~\cite{multiproc1} . By altering each one of the faceted values that contains both high level(confidential) and low level(public) information, a single process simulates the two processes of multi-process execution. The main advantage here being able to execute single execution that is the two mimicked executions collapses to a single one if both the values in a given faceted value are same and thus lessening the program overhead. More on faceted values can be see in the following chapters.

There has been a couple of papers on faceted value approach  ~\cite{bib4} ~\cite{bib3} to dynamic information flow analysis. This paper concentrates on properly handling exceptions with faceted approach. Chapter 2 describes some background information on types of information flow analysis and also shows some of the JavaScript attacks. Chapter 3 gives an introduction on basic faceted evaluation, its semantics followed by some of the scenarios where there is a need to handle exceptions and a theoretical explanation of handling exceptions using the language constructs defined in earlier paper ~\cite{bib4}. Chapter 4 and 5 takes you through the implementation part of the project with some of the examples from real time scenarios and how the  new feature has been embedded into Mozilla Firefox browser. Chapter 6 compares the performance of faceted value implementation to that of Secure-Multi execution and chapter 7 gives the conclusion.